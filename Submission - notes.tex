\documentclass[a4paper,twoside]{article}

\usepackage{epsfig}
\usepackage{subcaption}
\usepackage{calc}
\usepackage{amssymb}
\usepackage{amstext}
\usepackage{amsmath}
\usepackage{amsthm}
\usepackage{multicol}
\usepackage{pslatex}
\usepackage{apalike}
\usepackage{algorithm2e}
\usepackage[bottom]{footmisc}
\usepackage{SCITEPRESS}     % Please add other packages that you may need BEFORE the SCITEPRESS.sty package.

\begin{document}

\title{VERIAL: Verification-Enabled Runtime Integrity Attestation of Linux}

\author{\authorname{Michael Neises \orcidAuthor{0000-0002-5464-3269}}
\email{neisesmichael@gmail.com}
}

\keywords{Runtime Attestation, Remote Attestation, Virtual Machine Introspection, seL4, Operating Systems, Integrity, Confidentiality.}

\abstract{Runtime attestation is a way to gain confidence in the current state of a remote target. That confidence is valuable in safety-critical systems such as those involving
health-care or finance. Layered attestation is a way of extending that confidence from
one component to another. Solutions for layered attestation require strict isolation.
Without that isolation, the boundaries between components are blurred, and trustworthy components become intermingled with untrustworthy ones. The seL4 is uniquely
well-suited to offer kernel properties sufficient to achieve such isolation. While not
the world's only kernel with some formal verification, the seL4 is significantly more
advanced than any other solution while uniquely offering the possibility of real-time
computation. I design, implement, and evaluate introspective measurements and the
layered runtime attestation of a Linux kernel hosted by the seL4. VERIAL can detect
diamorphine-style rootkits with performance cost comparable to previous work.}

\onecolumn \maketitle \normalsize \setcounter{footnote}{0} \vfill

\section{\uppercase{Introduction}}
\label{sec:introduction}

Your paper will be part of the conference proceedings
therefore we ask that authors follow the guidelines explained in
this example in order to achieve the highest quality possible
\cite{Smith98}.

Be advised that papers in a technically unsuitable form will be
returned for retyping. After returned the manuscript must be
appropriately modified.

\section{\uppercase{Manuscript Preparation}}

We strongly encourage authors to use this document for the
preparation of the camera-ready. Please follow the instructions
closely in order to make the volume look as uniform as possible
\cite{Moore99}.

\subsection{First Section}

This section must be in one column.

\subsubsection{Title and Subtitle}

Use the command \textit{$\backslash$title} and follow the given structure in "example.tex". The title and subtitle must be with initial letters
capitalized (titlecased). The separation between the title and subtitle is done by adding a colon ":" just before the subtitle beginning. In the title or subtitle, words like "is", "or", "then", etc. should not be capitalized unless they are the first word of the title or subtitle. No formulas or special characters of any form or language are allowed in the title or subtitle.

\subsection{Second Section}

Files "example.tex" and "example.bib" show how to create a paper
with a corresponding list of references.

\subsubsection{Section Titles}

The heading of a section title should be in all-capitals.

Example: \textit{$\backslash$section\{FIRST TITLE\}}

\subsubsection{Tables}

Tables must appear inside the designated margins or they may span
the two columns.

Tables in two columns must be positioned at the top or bottom of the
page within the given margins. To span a table in two columns please add an asterisk (*) to the table \textit{begin} and \textit{end} command.

Example: \textit{$\backslash$begin\{table*\}}

\hspace*{1.5cm}\textit{$\backslash$end\{table*\}}\\

Tables should be centered and should always have a caption
positioned above it. The font size to use is 9-point. No bold or
italic font style should be used.

The final sentence of a caption should end with a period.

\begin{table}[h]
\caption{This caption has one line so it is
centered.}\label{tab:example1} \centering
\begin{tabular}{|c|c|}
  \hline
  Example column 1 & Example column 2 \\
  \hline
  Example text 1 & Example text 2 \\
  \hline
\end{tabular}
\end{table}

\begin{table}[h]
\vspace{-0.2cm}
\caption{This caption has more than one line so it has to be
justified.}\label{tab:example2} \centering
\begin{tabular}{|c|c|}
  \hline
  Example column 1 & Example column 2 \\
  \hline
  Example text 1 & Example text 2 \\
  \hline
\end{tabular}
\end{table}

Please note that the word "Table" is spelled out.


\subsubsection{Figures}

Figures must appear inside the designated margins or they may span
the two columns.

Figures in two columns must be positioned at the top or bottom of
the page within the given margins. To span a figure in two columns please add an asterisk (*) to the figure \textit{begin} and \textit{end} command.

Example: \textit{$\backslash$begin\{figure*\}}

\hspace*{1.5cm}\textit{$\backslash$end\{figure*\}}

Figures should be centered and should always have a caption
positioned under it. The font size to use is 9-point. No bold or
italic font style should be used.

\begin{figure}[!h]
  \centering
   {\epsfig{file = SCITEPRESS.eps, width = 5.5cm}}
  \caption{This caption has one line so it is centered.}
  \label{fig:example1}
 \end{figure}

\begin{figure}[!h]
  \vspace{-0.2cm}
  \centering
   {\epsfig{file = SCITEPRESS.eps, width = 5.5cm}}
  \caption{This caption has more than one line so it has to be justified.}
  \label{fig:example2}
\end{figure}

The final sentence of a caption should end with a period.



Please note that the word "Figure" is spelled out.

\subsubsection{Equations}

Equations should be placed on a separate line, numbered and
centered.\\The numbers accorded to equations should appear in
consecutive order inside each section or within the contribution,
with the number enclosed in brackets and justified to the right,
starting with the number 1.

Example:

\begin{equation}\label{eq1}
    a=b+c
\end{equation}

\subsubsection{Algorithms and Listings}

Algorithms and Listings captions should have font size 9-point, no bold or
italic font style should be used and the final sentence of a caption should end with a period. The separator between the title of algorithms/listings and the name of the algorithm/listing must be a colon.
Captions with one line should be centered and if it has more than one line it should be set to justified.

\begin{algorithm}[!h]
 \caption{How to write algorithms.}
 \KwData{this text}
 \KwResult{how to write algorithm with \LaTeX2e }
 initialization\;
 \While{not at end of this document}{
  read current\;
  \eIf{understand}{
   go to next section\;
   current section becomes this one\;
   }{
   go back to the beginning of current section\;
  }
 }
\end{algorithm}


\bigskip
\subsubsection{Program Code}\label{subsubsec:program_code}

Program listing or program commands in text should be set in
typewriter form such as Courier New.

Example of a Computer Program in Pascal:

\begin{small}
\begin{verbatim}
 Begin
     Writeln('Hello World!!');
 End.
\end{verbatim}
\end{small}


The text must be aligned to the left and in 9-point type.

\section{\uppercase{Conclusions}}
\label{sec:conclusion}

We hope you find the information in this template useful in the preparation of your submission.

\bibliographystyle{apalike}
{\small
\bibliography{example}}

\section*{\uppercase{Appendix}}

If any, the appendix should appear directly after the
references without numbering, and not on a new page. To do so please use the following command:
\textit{$\backslash$section*\{APPENDIX\}}

\end{document}

